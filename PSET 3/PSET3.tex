\documentclass{article}
\usepackage[usenames, dvipsnames]{color}
\usepackage{fancyhdr}
\usepackage{extramarks}
\usepackage{amsmath}
\usepackage{amssymb}
\usepackage{mathtools}
\usepackage{amsthm}
\usepackage{amsfonts}
\usepackage[per-mode=fraction]{siunitx}
\usepackage{tikz}
\usepackage{graphicx}
\usepackage{float}
\usepackage{afterpage}
\usepackage{systeme}
\usepackage{makecell}
\usepackage{nicematrix}
\usepackage{physics}
\usepackage{float}
\usepackage{tikz}
\usetikzlibrary{math,arrows,positioning,shapes,fit,calc}
\usepackage{dsfont}
\usepackage{wrapfig}
\usepackage{pgfplots}
\usepackage{esint}
\pgfplotsset{compat=1.8}

\definecolor{mediumgreen}{RGB}{0,153,0}
\newtheorem{theorem}{Theorem}[section]
\newtheorem{lemma}[theorem]{Lemma}
\usetikzlibrary{trees}

%
% Basic Document Settings
%

\topmargin=-0.45in
\evensidemargin=0in
\oddsidemargin=0in
\textwidth=6.5in
\textheight=9.0in
\headsep=0.25in

\linespread{1.1}

\pagestyle{fancy}
\lhead{}
\chead{\hmwkClass\ (\hmwkClassInstructor\/): \hmwkTitle}
\rhead{\firstxmark}
\lfoot{\lastxmark}
\cfoot{\thepage}

\renewcommand\headrulewidth{0.4pt}
\renewcommand\footrulewidth{0.4pt}

\setlength\parindent{0pt}

%
% Create Problem Sections
%

\newcommand{\enterProblemHeader}[1]{
    \nobreak\extramarks{}{Problem \arabic{#1} continued on next page\ldots}\nobreak{}
    \nobreak\extramarks{Problem \arabic{#1} (continued)}{Problem \arabic{#1} continued on next page\ldots}\nobreak{}
}

\newcommand{\exitProblemHeader}[1]{
    \nobreak\extramarks{Problem \arabic{#1} (continued)}{Problem \arabic{#1} continued on next page\ldots}\nobreak{}
    \stepcounter{#1}
    \nobreak\extramarks{Problem \arabic{#1}}{}\nobreak{}
}

\setcounter{secnumdepth}{0}
\newcounter{partCounter}
\newcounter{homeworkProblemCounter}
\setcounter{homeworkProblemCounter}{1}
\nobreak\extramarks{Problem \arabic{homeworkProblemCounter}}{}\nobreak{}

%
% Homework Problem Environment
%
% This environment takes an optional argument. When given, it will adjust the
% problem counter. This is useful for when the problems given for your
% assignment aren't sequential. See the last 3 problems of this template for an
% example.
%
\newenvironment{homeworkProblem}[1][-1]{
    \ifnum#1>0
        \setcounter{homeworkProblemCounter}{#1}
    \fi
    \section{Problem \arabic{homeworkProblemCounter}}
    \setcounter{partCounter}{1}
    \enterProblemHeader{homeworkProblemCounter}
}{
    \exitProblemHeader{homeworkProblemCounter}
}

%
% Homework Details
%   - Title
%   - Due date
%   - Class
%   - Section/Time
%   - Instructor
%   - Author
%

\newcommand{\hmwkTitle}{Homework 3}
\newcommand{\hmwkDueDate}{Monday January 24, 2022}
\newcommand{\hmwkClass}{Math 382}
\newcommand{\hmwkClassTime}{Section A}
\newcommand{\hmwkClassInstructor}{Prof. Ezra Getzler}
\newcommand{\hmwkAuthorName}{\textbf{Anthony Tam}}

%
% Title Page
%

\title{
    \vspace{2in}
    \textmd{\textbf{\hmwkClass:\ \hmwkTitle}}\\
    \normalsize\vspace{0.1in}\small{Due\ on\ \hmwkDueDate\ at 5:00 PM}\\
    \vspace{0.1in}\large{\textit{\hmwkClassInstructor\ }}
    \vspace{3in}
}

\author{\hmwkAuthorName}
\date{}

\renewcommand{\part}[1]{\textbf{\large Part \Alph{partCounter}}\stepcounter{partCounter}\\}

%
% Various Helper Commands
%

% For derivatives
\newcommand{\deriv}[1]{\frac{\mathrm{d}}{\mathrm{d}x} (#1)}

% For partial derivatives
\newcommand{\pderiv}[2]{\frac{\partial}{\partial #1} (#2)}

% Integral dx
\newcommand{\dx}{\mathrm{d}x}

% Alias for the Solution section header
\newcommand{\solution}{\textbf{\large Solution}}

% Probability commands: Expectation, Variance, Covariance, Bias
\newcommand{\E}{\mathrm{E}}
\newcommand{\Var}{\mathrm{Var}}
\newcommand{\Cov}{\mathrm{Cov}}
\newcommand{\Bias}{\mathrm{Bias}}

\newcommand*\circled[1]{\tikz[baseline=(char.base)]{
            \node[shape=circle,draw,inner sep=2pt] (char) {#1};}}

\newcommand\myeqq{\stackrel{\mathclap{\normalfont\mbox{\scriptsize\text{set}}}}{=}}

\makeatother

\begin{document}

\maketitle

\pagebreak

\begin{homeworkProblem}
\textbf{Gamelin \textsection 3.1 Exercise 7}: Show that the formula in Green's theorem is invariant under coordinate changes. Suppose that the theorem holds for a bounded domain $D$ with piecewise smooth boundary $\partial D=\gamma .$ Let $F(s, t)=(x(s, t), y(s, t))$ be a continuous function that maps $D$ smoothly, one-to-one and onto a bounded domain $E$, and the boundary $\gamma$ piecewise differentiably, one-to-one and onto the boundary $\eta$ of $E$. Suppose that the Jacobian
$$
J_{F}(s, t)=x_{s} y_{t}-x_{t} y_{s}>0
$$
is positive. Then Green's theorem holds for $E$. By a smooth function, we mean a function with continuous partial derivatives.\\

\solution

 \begin{proof}
   Suppose Green's theorem holds on $D \subseteq \mathbb{C}$, i.e., if $U$ and $V$ are $C^1$ functions of $s$ and $t$ on $\overline{D}$, then
   \begin{align*}
    \int_{\partial D = \gamma} U(s,t) ds + V(s,t) dt = \iint_{D} \left( \frac{\partial V}{\partial s} - \frac{\partial U}{\partial t}\right)dsdt
    .\end{align*}
   We want to show this holds for the region $E$, which is the image of $D$ under the coordinate transformation $F(s,t)$. First note that $F$ is smooth, i.e., is $C^{\infty}$. Now consider the double integral of the $C^1$ function $-P_y(x,y)$ on $F(D) = E$, and rewrite it as a double integral over $D$ by pulling back the integrand:
   \begin{align*}
    \iint_{F(D) = E} -P_y(x,y) dxdy = \iint_D (-P_y\circ F)(s,t) \abs{\text{det}\, DF(s,t)}dsdt
    .\end{align*}
    Since $\text{det}\, DF(s,t) = J_F(s,t) > 0$, then $F$ is orientation preserving and we can take out absolute values to get
    \begin{align*}
      \iint_{F(D) = E} -P_y(x,y) dxdy = \iint_D -P_y(x(s,t), y(s,t)) \left( \frac{\partial x}{\partial s}\frac{\partial y}{\partial t} - \frac{\partial x}{\partial t}\frac{\partial y}{\partial s} \right)dsdt
     .\end{align*}
   Now similarly for the $C^1$ function $Q_x(x,y)$ on $E$, we have
   \begin{align*}
     \iint_{F(D) = E} Q_x(x,y) dxdy = \iint_D Q_x(x(s,t), y(s,t)) \left( \frac{\partial x}{\partial s}\frac{\partial y}{\partial t} - \frac{\partial x}{\partial t}\frac{\partial y}{\partial s} \right)dsdt
    .\end{align*}
   Computing the line integral of $P$ over $\eta$, which is exactly the image of $\partial D = \gamma$ under $F$, we can use the change of variables formula again to get
   \begin{align*}
    \int_{F(\gamma) = \eta}P(x,y) dx &= \int_{\gamma} (P \circ F)(s,t) \left( \frac{\partial x}{\partial s}ds + \frac{\partial x}{\partial t}dt \right)\\
    &= \int_{\gamma} \left( P(x(s,t), y(s,t)) \frac{\partial x}{\partial s}ds + P(x(s,t), y(s,t)) \frac{\partial x}{\partial t}dt \right)
    .\end{align*}
   By noting that $\gamma = \partial D$ and that the integrand is indeed $C^1$, we can apply Green's Theorem to change this to a double integral on $D$ to get
   \begin{align*}
     \int_{\eta}P(x,y) dx &= \int_{\gamma} \left( P(x(s,t), y(s,t)) \frac{\partial x}{\partial s}ds + P(x(s,t), y(s,t)) \frac{\partial x}{\partial t}dt \right)\\
     &= \iint_D \partial_s \left( P(x(s,t), y(s,t)) \frac{\partial x}{\partial t} \right) - \partial_t \left( P(x(s,t), y(s,t)) \frac{\partial x}{\partial s}\right) dsdt\\
     &= \iint_D \left(\partial_s \left[ P(x(s,t), y(s,t)) \right]\frac{\partial x}{\partial t} + P(x(s,t), y(s,t))\frac{\partial^2 x}{\partial t\partial s}\right. \\
     & \hspace{1.5cm} \left. - \partial_t \left[ P(x(s,t), y(s,t)) \right] \frac{\partial x}{\partial s} - P(x(s,t), y(s,t))\frac{\partial^2 x}{\partial s\partial t}\right) dsdt
    .\end{align*}
   By the chain rule, we can compute
   \begin{align*}
     \partial_s \left[ P(x(s,t), y(s,t)) \right] = \frac{\partial P}{\partial x} \frac{\partial x}{\partial s} + \frac{\partial P}{\partial y} \frac{\partial y}{\partial s} \; \text{ and } \; \partial_t \left[ P(x(s,t), y(s,t)) \right] = \frac{\partial P}{\partial x} \frac{\partial x}{\partial t} + \frac{\partial P}{\partial y} \frac{\partial y}{\partial t}
    .\end{align*}
   Since $F$ is smooth and thus the coordinate functions are at least $C^2$, the mixed partials commute and cancel to give
   \begin{align*}
     \int_{\eta}P(x,y) dx &= \iint_D \left(\frac{\partial P}{\partial x} \frac{\partial x}{\partial s} + \frac{\partial P}{\partial y} \frac{\partial y}{\partial s}\right) \frac{\partial x}{\partial t} - \left(\frac{\partial P}{\partial x} \frac{\partial x}{\partial t} + \frac{\partial P}{\partial y} \frac{\partial y}{\partial t}\right) \frac{\partial x}{\partial s} dsdt\\
     &= \iint_D \left(\frac{\partial P}{\partial y} \frac{\partial y}{\partial s}\frac{\partial x}{\partial t} - \frac{\partial P}{\partial y} \frac{\partial y}{\partial t}\frac{\partial x}{\partial s}\right) dsdt\\
     &= \iint_D -P_y(x(s,t), y(s,t)) \left(\frac{\partial x}{\partial s}\frac{\partial y}{\partial t} - \frac{\partial x}{\partial t}\frac{\partial y}{\partial s}\right) dsdt
    .\end{align*}
   But recall above that we showed that this integral over $D$ is equivalent to another integral over $E$ by change of variables and hence
   \begin{align*}
     \int_{\eta}P(x,y) dx &= \iint_D -P_y(x(s,t), y(s,t)) \left(\frac{\partial x}{\partial s}\frac{\partial y}{\partial t} - \frac{\partial x}{\partial t}\frac{\partial y}{\partial s}\right) dsdt\\
     &= \iint_D (-P_y \circ F)(s,t) \, \text{det}\, DF(s,t)\, dsdt\\
     &= \iint_{F(D) = E} -P_y(x,y) dxdy
    ,\end{align*}
   which is one part of the statement of Green's Theorem. What is left to show is the statement about $Qdy$, so similarly compute
   \begin{align*}
     \int_{\eta}Q(x,y) dy &= \int_{\gamma} (Q \circ F)(s,t) \left( \frac{\partial y}{\partial s}ds + \frac{\partial y}{\partial t}dt \right)\\
     &= \int_{\gamma} \left( Q(x(s,t), y(s,t)) \frac{\partial y}{\partial s}ds + Q(x(s,t), y(s,t)) \frac{\partial y}{\partial t}dt \right)\\
     &= \iint_D \partial_s \left( Q(x(s,t), y(s,t)) \frac{\partial y}{\partial t} \right) - \partial_t \left( Q(x(s,t), y(s,t)) \frac{\partial y}{\partial s}\right) dsdt\\
     &= \iint_D \left(\partial_s \left[ Q(x(s,t), y(s,t)) \right]\frac{\partial y}{\partial t} + Q(x(s,t), y(s,t))\frac{\partial^2 y}{\partial t\partial s}\right. \\
     & \hspace{1.5cm} \left. - \partial_t \left[ Q(x(s,t), y(s,t)) \right] \frac{\partial y}{\partial s} - Q(x(s,t), y(s,t))\frac{\partial^2 y}{\partial s\partial t}\right) dsdt\\
     &= \iint_D \left(\frac{\partial Q}{\partial x} \frac{\partial x}{\partial s} + \frac{\partial Q}{\partial y} \frac{\partial y}{\partial s}\right) \frac{\partial y}{\partial t} - \left(\frac{\partial Q}{\partial x} \frac{\partial x}{\partial t} + \frac{\partial Q}{\partial y} \frac{\partial y}{\partial t}\right) \frac{\partial y}{\partial s} dsdt\\
     &= \iint_D Q_x(x(s,t), y(s,t)) \left(\frac{\partial x}{\partial s}\frac{\partial y}{\partial t} - \frac{\partial x}{\partial t}\frac{\partial y}{\partial s}\right) dsdt\\
     &= \iint_{D} (Q_x \circ F)(s,t)\, \text{det}\, DF(s,t)\, dsdt\\
     &= \iint_{F(D) = E} Q_x(x,y)\, dxdy
    .\end{align*}
   Adding both equalities gives the statement of Green's theorem on $E$,
   \begin{align*}
     \int_{\partial E = \eta}P(x,y) dx + Q(x,y) dy = \iint_{E} \left( \frac{\partial Q}{\partial x} - \frac{\partial P}{\partial y} \right) \, dxdy
    ,\end{align*}
  as claimed.
 \end{proof}
\end{homeworkProblem}

\pagebreak

\begin{homeworkProblem}
The Fundamental Theorem of Calculus says that
$$
F\left(z_{1}\right)-F\left(z_{0}\right)=\int_{\gamma} f(z) d z
$$
where $F(z)$ is a complex differentiable function with domain $U, \gamma$ is a piecewise differentiable function in $U$ starting at $z_{0}$ and ending at $z_{1}$, and $f(z)=F^{\prime}(z)$. Write out the proof of the special case that $\gamma$ is a path obtained by concatenating together a finite number of paths that are parallel to the axes. As we saw in class, by Cauchy's Theorem, this implies the Fundamental Theorem of Calculus in the general case.\\

\solution

 \begin{proof}
Suppose $F: U \to \mathbb{C}$ with domain $\varnothing \neq U \subseteq \mathbb{C}$ be a primitive for $f$ with $F(z) = F(x+iy) = u(x,y) + i v(x,y)$. Let $C$ be a smooth curve with $\gamma: [a,b] \to U$ an orientation preserving parameterization defined by $\gamma(t) = x(t) + iy(t)$. Let us first recall that from Discussion 2 that a $C^1$ function defined on $U$ is complex differentiable at $z$ if and only if its derivative at $z$ considered as a function from $\mathbb{R}^2 \to \mathbb{R}^2$ is of the form $\begin{bmatrix} a & - b \\ b & a \end{bmatrix}$ for some $a, b \in \mathbb{R}$. Further, we worked out that this condition was satisfied by the Cauchy Riemann equations. So, using this fact, consider the one-to-one correspondances
\begin{align*}
 F(z) = u(x,y) + iv(x,y) \, \longleftrightarrow \mathbf{F}(x,y) = \begin{bmatrix} u(x,y) \\ v(x,y) \end{bmatrix}\\
 \gamma(t) = x(t) + iy(t) \, \longleftrightarrow \mathbf{\Gamma}(t) = \begin{bmatrix} x(t) \\ y(t) \end{bmatrix}
 .\end{align*}
We can abuse notation and represent the derivative of $F$ and its compositions with $\gamma$ in matrix form under these one to one correspondances and utilize the multivariable chain rule to compute
\begin{align*}
 F'(\gamma(t))\gamma'(t) &\leftrightarrow D \mathbf{F}(\mathbf{\Gamma}(t))D \mathbf{\Gamma}(t)\\
 &= \left[\begin{matrix} \frac{\partial u}{\partial x} & \frac{\partial u}{\partial y}\\ \frac{\partial v}{\partial x} & \frac{\partial v}{\partial y} \end{matrix}\right] \left[\begin{matrix} \frac{\partial x}{\partial t} \\ \frac{\partial y}{\partial t} \end{matrix}\right]\\
 &= \begin{bmatrix} \frac{\partial u}{\partial x}\frac{\partial x}{\partial t} + \frac{\partial u}{\partial y}\frac{\partial y}{\partial t} \\ \frac{\partial v}{\partial x}\frac{\partial x}{\partial t} +  \frac{\partial v}{\partial y}\frac{\partial y}{\partial t}\end{bmatrix}\\
 &= \begin{bmatrix} \frac{\partial u}{\partial t} \\ \frac{\partial v}{\partial t} \end{bmatrix}\\
 &= D( \mathbf{F} \circ \mathbf{\Gamma})(t)\\
 &\leftrightarrow D(F \circ \gamma)(t)\\
 &= (F\circ \gamma)'(t)
 .\end{align*}
Using this result, we can safely compute the desire line integral:
\begin{align*}
 \int_C f(z) dz &= \int_{C} F'(z) dz \\
 &= \int_a^b F'(\gamma(t))\gamma(t) dt\\
 &= \int_a^b (F \circ \gamma)'(t) dt\\
 &= \int_a^b \text{Re}\left[(F \circ \gamma)'(t)\right] dt + i \int_a^b \text{Im}\left[(F \circ \gamma)'(t)\right] dt,\end{align*}
where we split up real and imaginary parts. Since differentiation is done coordinate-wise, we can bring in the real and imaginary part operators and apply the single-variable real-valued Fundamental Theorem of Calculus:
\begin{align*}
  \int_C f(z) dz &= \int_a^b (\text{Re}\, F \circ \gamma)'(t) dt + i \int_a^b (\text{Im}\,F \circ \gamma)'(t)dt\\
  &= \text{Re}\, F(\gamma(b)) - \text{Re}\, F(\gamma(a)) + i \,\text{Im}\, F(\gamma(b)) - i\, \text{Im}\, F(\gamma(a))\\
  &= \left[\text{Re}\, F(\gamma(b))+ i \,\text{Im}\, F(\gamma(b))\right] - \left[ \text{Re}\, F(\gamma(a)) + i\, \text{Im}\, F(\gamma(a)) \right]\\
  &= F(\gamma(b)) - F(\gamma(a))\\
  &= F(z_1) - F(z_0)
 .\end{align*}
If $C = C_1 \cup \cdots \cup C_k$ is piecewise smooth, then denote by $a_i$ and $b_i$ the starting and ending points of $C_i$, and note that $a_1 = z_0$, $b_k = z_1$, and $b_i = a_{i + 1}$ for each $i = 1, \ldots, k -1$. Thus we have the same result for piecewise smooth curves
\begin{align*}
 \int_C f(z)dz = \int_{C} F'(z) dz &= \int_{C_k}F'(z)dz + \cdots + \int_{C_1} F'(z)dz\\
 &= F(b_k) \underbrace{- F(a_k) + F(b_{k-1})}_{=0} \underbrace{- F(a_{k-1}) + F(b_{k-2})}_{=0} - \cdots \underbrace{- F(a_2) + F(b_1)}_{=0}- F(a_1)\\
 &= F(z_1) - F(z_0)
 ,\end{align*}
as claimed.
 \end{proof}
\end{homeworkProblem}

\pagebreak

\begin{homeworkProblem}
\textbf{Gamelin \textsection IV.3 Exercise 1}:
By taking the line integral of the complex differentiable function $f(z)=e^{-z^{2} / 2}$ around a rectangle with vertices $\pm R$, it $\pm R$, and sending $R$ to $\infty$, show that
$$
\frac{1}{\sqrt{2 \pi}} \int_{-\infty}^{\infty} e^{-x^{2} / 2} e^{-i t x} d x=e^{-t^{2} / 2}, \quad-\infty<t<\infty
$$
You may assume the definite integral
$$
\int_{-\infty}^{\infty} e^{-x^{2} / 2} d x=\sqrt{2 \pi}.
$$

\solution

\begin{proof}
  Let $C$ be the curve traversing the rectangle with vertices $\left\{ \pm R, it \pm R \right\}$. Note that $C$ is a simple, closed curve so the following closed loop integral vanishes
  \begin{align*}
    \oint_C e^{-z^2/2} = 0
   ,\end{align*}
  which follows from the fact that $f(z) = e^{-z^2}$ is analytic and $C$ lies in $\mathbb{C}$, a simply connected domain, and applying Cauchy's Theorem. But, let us parametrize $C$ by the following piecewise differentiable paths
  \begin{align*}
   \begin{cases}
    \gamma_1(s) = s, & s \in [-R, R]\\
    \gamma_2(s) = R + is, & s \in [0,t]\\
    \gamma_3(s) = -s + it, & s \in [-R, R]\\
    \gamma_4(s) = R + (t - s)i, & s \in [0,t],
   \end{cases}
   \end{align*}
  such that $C = \gamma_1 \cup \gamma_2 \cup \gamma_3 \cup \gamma_4$. Then we have the following relation,
  \begin{align*}
   0 = \oint_C e^{-z^2/2} = \int_{-R}^{R} e^{-s^2/2}ds + i \int_{0}^t e^{-(R+is)^2/2}ds - \int_{-R}^R e^{-(-s + it)^2/2}ds - i \int_{0}^te^{-(R + (t-s)i)^2/2}ds
   .\end{align*}
  But note that the in two integrals from $0$ to $t$, we have $R$ dependence in the integrand so that as we take the limit as $R$ tends to $\infty$, we get that for any fixed $t$, the integral vanishes:
  \begin{align*}
    i \int_{0}^t e^{-(R+is)^2/2}ds = i \int_0^t e^{-R^2/2}e^{-isR}e^{s^2/2}ds = 0
   ,\end{align*}
  since the $e^{-R^2/2}$ term will dominate to zero as $R \to \infty $ as the $e^{-isR}e^{s^2/2}$ term stays bounded; the integral $- i \int_{0}^te^{-(R + (t-s)i)^2/2}ds$ similarly vanishes. So overall after taking limits, we have
  \begin{align*}
    0 = \lim_{R \to \infty}\int_{-R}^{R} e^{-s^2/2}ds - \lim_{R \to \infty}\int_{-R}^R e^{-(-s + it)^2/2}ds
   ,\end{align*}
  which after expanding and using the value of the Gaussian integral, we get
  \begin{align*}
    \lim_{R \to \infty}\int_{-R}^{R} e^{-s^2/2}ds = \lim_{R \to \infty}\int_{-R}^R e^{-(-s + it)^2/2}ds\\
    \iff \sqrt{2\pi} = e^{t^2/2} \int_{-\infty}^{\infty}e^{-s^2/2}e^{-ist}ds\\
    \iff \frac{1}{\sqrt{2\pi}} \int_{-\infty}^{\infty}e^{-s^2/2}e^{-ist}ds = e^{-t^2/2}
   ,\end{align*}
  as was to be shown.
\end{proof}
\end{homeworkProblem}

\end{document}
