\documentclass{article}
\usepackage[usenames, dvipsnames]{color}
\usepackage{fancyhdr}
\usepackage{extramarks}
\usepackage{amsmath}
\usepackage{amssymb}
\usepackage{mathtools}
\usepackage{amsthm}
\usepackage{amsfonts}
\usepackage[per-mode=fraction]{siunitx}
\usepackage{tikz}
\usepackage{graphicx}
\usepackage{float}
\usepackage{afterpage}
\usepackage{systeme}
\usepackage{makecell}
\usepackage{nicematrix}
\usepackage{physics}
\usepackage{float}
\usepackage{tikz}
\usetikzlibrary{math,arrows,positioning,shapes,fit,calc}
\usepackage{dsfont}
\usepackage{wrapfig}
\usepackage{pgfplots}
\usepackage{esint}
\pgfplotsset{compat=1.8}

\definecolor{mediumgreen}{RGB}{0,153,0}
\newtheorem{theorem}{Theorem}[section]
\newtheorem{lemma}[theorem]{Lemma}
\usetikzlibrary{trees}

%
% Basic Document Settings
%

\topmargin=-0.45in
\evensidemargin=0in
\oddsidemargin=0in
\textwidth=6.5in
\textheight=9.0in
\headsep=0.25in

\linespread{1.1}

\pagestyle{fancy}
\lhead{}
\chead{\hmwkClass\ (\hmwkClassInstructor\/): \hmwkTitle}
\rhead{\firstxmark}
\lfoot{\lastxmark}
\cfoot{\thepage}

\renewcommand\headrulewidth{0.4pt}
\renewcommand\footrulewidth{0.4pt}

\setlength\parindent{0pt}

%
% Create Problem Sections
%

\newcommand{\enterProblemHeader}[1]{
    \nobreak\extramarks{}{Problem \arabic{#1} continued on next page\ldots}\nobreak{}
    \nobreak\extramarks{Problem \arabic{#1} (continued)}{Problem \arabic{#1} continued on next page\ldots}\nobreak{}
}

\newcommand{\exitProblemHeader}[1]{
    \nobreak\extramarks{Problem \arabic{#1} (continued)}{Problem \arabic{#1} continued on next page\ldots}\nobreak{}
    \stepcounter{#1}
    \nobreak\extramarks{Problem \arabic{#1}}{}\nobreak{}
}

\setcounter{secnumdepth}{0}
\newcounter{partCounter}
\newcounter{homeworkProblemCounter}
\setcounter{homeworkProblemCounter}{1}
\nobreak\extramarks{Problem \arabic{homeworkProblemCounter}}{}\nobreak{}

%
% Homework Problem Environment
%
% This environment takes an optional argument. When given, it will adjust the
% problem counter. This is useful for when the problems given for your
% assignment aren't sequential. See the last 3 problems of this template for an
% example.
%
\newenvironment{homeworkProblem}[1][-1]{
    \ifnum#1>0
        \setcounter{homeworkProblemCounter}{#1}
    \fi
    \section{Problem \arabic{homeworkProblemCounter}}
    \setcounter{partCounter}{1}
    \enterProblemHeader{homeworkProblemCounter}
}{
    \exitProblemHeader{homeworkProblemCounter}
}

%
% Homework Details
%   - Title
%   - Due date
%   - Class
%   - Section/Time
%   - Instructor
%   - Author
%

\newcommand{\hmwkTitle}{Homework 4}
\newcommand{\hmwkDueDate}{Sunday January 30, 2022}
\newcommand{\hmwkClass}{Math 382}
\newcommand{\hmwkClassTime}{Section A}
\newcommand{\hmwkClassInstructor}{Prof. Ezra Getzler}
\newcommand{\hmwkAuthorName}{\textbf{Anthony Tam}}

%
% Title Page
%

\title{
    \vspace{2in}
    \textmd{\textbf{\hmwkClass:\ \hmwkTitle}}\\
    \normalsize\vspace{0.1in}\small{Due\ on\ \hmwkDueDate\ at 5:00 PM}\\
    \vspace{0.1in}\large{\textit{\hmwkClassInstructor\ }}
    \vspace{3in}
}

\author{\hmwkAuthorName}
\date{}

\renewcommand{\part}[1]{\textbf{\large Part \Alph{partCounter}}\stepcounter{partCounter}\\}

%
% Various Helper Commands
%

% For derivatives
\newcommand{\deriv}[1]{\frac{\mathrm{d}}{\mathrm{d}x} (#1)}

% For partial derivatives
\newcommand{\pderiv}[2]{\frac{\partial}{\partial #1} (#2)}

% Integral dx
\newcommand{\dx}{\mathrm{d}x}

% Alias for the Solution section header
\newcommand{\solution}{\textbf{\large Solution}}

% Probability commands: Expectation, Variance, Covariance, Bias
\newcommand{\E}{\mathrm{E}}
\newcommand{\Var}{\mathrm{Var}}
\newcommand{\Cov}{\mathrm{Cov}}
\newcommand{\Bias}{\mathrm{Bias}}

\newcommand*\circled[1]{\tikz[baseline=(char.base)]{
            \node[shape=circle,draw,inner sep=2pt] (char) {#1};}}

\newcommand\myeqq{\stackrel{\mathclap{\normalfont\mbox{\scriptsize\text{set}}}}{=}}

\makeatother

\begin{document}

\maketitle

\pagebreak

\begin{homeworkProblem}
The function $f(z)=\tan z$ solves the first-order ordinary differential equation $f^{\prime}(z)=1+f(z)^{2}$, with initial condition $f(0)=0$.

1) Show that the solution of this equation is odd. (Hint: the solution of a first-order ordinary differential equation is determined by its value at $z=0$. But $g(z)=-f(-z)$ solves the same equation.)

2) Consider the Taylor series of $f(z)$ around $z=0$ :
$$
\sum_{k=0}^{\infty} \frac{a_{2 k+1} z^{2 k+1}}{(2 k+1) !}
$$
(This is the most general Taylor series for an odd function.) Convert the differential equation into an equation for $a_{2 k+1}$ in terms of $a_{2 j+1}, 0 \leq j<k$.

3) Write pseudocode for a program to calculate the coefficient $a_{2 k+1}$, given a positive integer $k$. This may be an outline of a computer program in any of your favorite structured languages (for example Python), together with a series of lines that explain the sequence of instructions in such a program.

4) Calculate the first three nonzero terms of the Taylor series. How accurate is the resulting polynomial as an approximation for $\tan z$, when $z=1 / 100 ?$

5) Calculate the first three nonzero coefficients of the power series $f(z) \cos z$, and show that they equal the first three nonzero coefficients of the power series $\sin z$.\\

\solution\\

\part

\vspace{-0.5cm}

 \begin{proof}
Note that if $f(z)$ solves the differential equation, the function $g(z) = -f(-z)$ solves it too:
\begin{align*}
 g'(z) = 1 + g(z)^2 \iff f'(-z) = 1 + [f(-z)]^2
 .\end{align*}
But, we know that linear combinations of particular solutions also give a solution, so in particular
\begin{align*}
 \frac{f(x) + g(x)}{2} = \frac{f(x)- f(-x)}{2}
 ,\end{align*}
is also a solution. Given the initial condition $f(0) = 0$, we know that this solution is unique, so we must have
\begin{align*}
 f(x) = \frac{f(x)- f(-x)}{2} \iff f(-x) = -f(x)
 ,\end{align*}
and thus the solution $f(z)$ is odd.
 \end{proof}

\part\\

Plugging in the Taylor series, we get the following after differentiating term by term and expanding the Cauchy product
\begin{align*}
 \left( \sum_{k=0}^{\infty} a_{2k+1} \frac{z^{2k + 1}}{(2k+1)!}\right)' = 1 + \left(\sum_{k=0}^{\infty} a_{2k+1} \frac{z^{2k + 1}}{(2k+1)!}\right)^2\\
 \iff \sum_{k=0}^{\infty} a_{2k+1} \frac{z^{2k}}{(2k)!} - \sum_{k=0}^{\infty} \left( \sum_{j=0}^{k} \frac{a_{2j + 1}}{(2j + 1)!} \cdot \frac{a_{2(k-j) + 1}}{(2(k-j) + 1)!} \right)z^{2k+1} - 1 = 0
 .\end{align*}
\textcolor{red}{FINISH}
\end{homeworkProblem}

\pagebreak

\begin{homeworkProblem}
  Calculate the Taylor series of the function $f_{k}(z)=(1-z)^{-k}$ at $z=0$. What is the radius of convergence of this series? Justify.\\

  \solution\\

  \vspace{-0.5cm}

   \begin{proof}
     First let us compute the $n^{\text{th}}$ derivative of $f_k$ at zero:
     \begin{align*}
      f_k^{(1)}(0) = k,\,  f_k^{(1)}(0) = k(k+1)\,, \ldots, f_k^{(n)}(0) = k(k+1)\cdots(k+n-1) = k^{(n)}
      ,\end{align*}
    where $k^{(n)}$ denotes the rising factorial. Then the Taylor series of $f_k(z)$ centered at $z=0$ is given by
    \begin{align*}
     f_k(z) = \sum_{n=0}^{\infty} \frac{f^{(n)}(0)}{n!}z^n = \sum_{n=0}^{\infty} \frac{k^{(n)}}{n!}z^n = \sum_{n=0}^{\infty} \left(\begin{matrix}k + n-1 \\ n\end{matrix}\right) z^n
     ,\end{align*}
    where we use the relation $k^{(n)}=\frac{(k+n-1) !}{(k-1) !}$ and thus $ \frac{k^{(n)}}{n!} = \left(\begin{matrix}k + n-1 \\ n\end{matrix}\right) $. For the radius of convergence, let us apply the ratio test to $a_k = \left(\begin{matrix}k + n-1 \\ n\end{matrix}\right)$:
    \begin{align*}
     R = \lim_{n \to \infty} \abs{ \frac{\left(\begin{matrix}k + n-1 \\ n\end{matrix}\right)}{\left(\begin{matrix}k + n \\ n +1 \end{matrix}\right)} } &= \lim_{n \to \infty} \frac{(k + n -1)!}{n! (k-1)!} \cdot \frac{(n+1)!(k+1)!}{(k+n)!}\\
     &= \lim_{n \to \infty} \frac{(k + n -1)!}{n! (k-1)!} \cdot \frac{(n+1)(n!)(k+1)(k)(k-1)!}{(k+n)(k+n-1)!}\\
     &= \lim_{n \to \infty} \frac{k(k+1)(n+1)}{k+n}\\
     &= k(k+1)
     .\end{align*}
    Hence, the radius of convergence is $R = k(k+1)$.
   \end{proof}
\end{homeworkProblem}

\pagebreak

\begin{homeworkProblem}
  \textbf{Gameline \textsection V.3 Exercise 6}: Show the series $\sum a_{k} z^{k}$, the differentiated series $\sum k a_{k} z^{k-1}$, and the integrated series $\sum \frac{a_{k}}{k+1} z^{k+1}$ all have the same radius of convergence.\\

  \solution\\

  \vspace{-0.5cm}

   \begin{proof}
   Consider the power series $\sum_{k \geq 0} a_k z^k$. The Cauchy Hadamard formula gives taht this power series has a radius of convergence
   \begin{align*}
    R = \frac{1}{\limsup_{k \to \infty} \sqrt[k]{\abs{a_k}}}
    .\end{align*}
    Using the fact that $\limsup_{k \to \infty} \sqrt[k]{k} = 1$, we have that the differentiated series $\sum_{k \geq 1} ka_k z^{k-1}$ has radius of convergence
    \begin{align*}
     R' &= \frac{1}{\limsup_{k \to \infty} \sqrt[k]{\abs{ka_k}}}\\
     &= \frac{1}{\limsup_{k \to \infty} \sqrt[k]{k}\sqrt[k]{\abs{a_k}}}\\
     &= \frac{1}{\limsup_{k \to \infty} \sqrt[k]{\abs{a_k}}}\\
     &= R
     .\end{align*}
     Similarly, using the fact that $\limsup_{k \to \infty} \sqrt[k]{1/(k+1)} = 1$, we have that the integrated series $\sum_{k \geq 0} \frac{a_k}{k+1}z^{k+1}$ has radius of convergence
     \begin{align*}
       R'' &= \frac{1}{\limsup_{k \to \infty} \sqrt[k]{\abs{ \frac{a_k}{k+1} }}}\\
       &= \frac{1}{\limsup_{k \to \infty} \sqrt[k]{\frac{1}{k+1} }\sqrt[k]{\abs{a_k}}}\\
       &= \frac{1}{\limsup_{k \to \infty} \sqrt[k]{\abs{a_k}}}\\
       &= R
      .\end{align*}
    Thus all series have the same radius of convergence $R = R' = R''$.
   \end{proof}
\end{homeworkProblem}

\end{document}
