\documentclass{article}
\usepackage[usenames, dvipsnames]{color}
\usepackage{fancyhdr}
\usepackage{extramarks}
\usepackage{amsmath}
\usepackage{amssymb}
\usepackage{mathtools}
\usepackage{amsthm}
\usepackage{amsfonts}
\usepackage[per-mode=fraction]{siunitx}
\usepackage{tikz}
\usepackage{graphicx}
\usepackage{float}
\usepackage{afterpage}
\usepackage{systeme}
\usepackage{makecell}
\usepackage{nicematrix}
\usepackage{physics}
\usepackage{float}
\usepackage{tikz}
\usetikzlibrary{math,arrows,positioning,shapes,fit,calc}
\usepackage{dsfont}
\usepackage{wrapfig}
\usepackage{pgfplots}
\usepackage{esint}
\usepackage{algorithm}
\usepackage{algorithmic}

\pgfplotsset{compat=1.8}

\definecolor{mediumgreen}{RGB}{0,153,0}
\newtheorem{theorem}{Theorem}[section]
\newtheorem{lemma}[theorem]{Lemma}
\usetikzlibrary{trees}

%
% Basic Document Settings
%

\topmargin=-0.45in
\evensidemargin=0in
\oddsidemargin=0in
\textwidth=6.5in
\textheight=9.0in
\headsep=0.25in

\linespread{1.1}

\pagestyle{fancy}
\lhead{}
\chead{\hmwkClass\ (\hmwkClassInstructor\/): \hmwkTitle}
\rhead{\firstxmark}
\lfoot{\lastxmark}
\cfoot{\thepage}

\renewcommand\headrulewidth{0.4pt}
\renewcommand\footrulewidth{0.4pt}

\setlength\parindent{0pt}

%
% Create Problem Sections
%

\newcommand{\enterProblemHeader}[1]{
    \nobreak\extramarks{}{Problem \arabic{#1} continued on next page\ldots}\nobreak{}
    \nobreak\extramarks{Problem \arabic{#1} (continued)}{Problem \arabic{#1} continued on next page\ldots}\nobreak{}
}

\newcommand{\exitProblemHeader}[1]{
    \nobreak\extramarks{Problem \arabic{#1} (continued)}{Problem \arabic{#1} continued on next page\ldots}\nobreak{}
    \stepcounter{#1}
    \nobreak\extramarks{Problem \arabic{#1}}{}\nobreak{}
}

\setcounter{secnumdepth}{0}
\newcounter{partCounter}
\newcounter{homeworkProblemCounter}
\setcounter{homeworkProblemCounter}{1}
\nobreak\extramarks{Problem \arabic{homeworkProblemCounter}}{}\nobreak{}

%
% Homework Problem Environment
%
% This environment takes an optional argument. When given, it will adjust the
% problem counter. This is useful for when the problems given for your
% assignment aren't sequential. See the last 3 problems of this template for an
% example.
%
\newenvironment{homeworkProblem}[1][-1]{
    \ifnum#1>0
        \setcounter{homeworkProblemCounter}{#1}
    \fi
    \section{Problem \arabic{homeworkProblemCounter}}
    \setcounter{partCounter}{1}
    \enterProblemHeader{homeworkProblemCounter}
}{
    \exitProblemHeader{homeworkProblemCounter}
}

%
% Homework Details
%   - Title
%   - Due date
%   - Class
%   - Section/Time
%   - Instructor
%   - Author
%

\newcommand{\hmwkTitle}{Homework 5}
\newcommand{\hmwkDueDate}{Wednesday February 9, 2022}
\newcommand{\hmwkClass}{Math 382}
\newcommand{\hmwkClassTime}{Section A}
\newcommand{\hmwkClassInstructor}{Prof. Ezra Getzler}
\newcommand{\hmwkAuthorName}{\textbf{Anthony Tam}}

%
% Title Page
%

\title{
    \vspace{2in}
    \textmd{\textbf{\hmwkClass:\ \hmwkTitle}}\\
    \normalsize\vspace{0.1in}\small{Due\ on\ \hmwkDueDate\ at 10:00 PM}\\
    \vspace{0.1in}\large{\textit{\hmwkClassInstructor\ }}
    \vspace{3in}
}

\author{\hmwkAuthorName}
\date{}

\renewcommand{\part}[1]{\textbf{\large Part \Alph{partCounter}}\stepcounter{partCounter}\\}

%
% Various Helper Commands
%

% For derivatives
\newcommand{\deriv}[1]{\frac{\mathrm{d}}{\mathrm{d}x} (#1)}

% For partial derivatives
\newcommand{\pderiv}[2]{\frac{\partial}{\partial #1} (#2)}

% Integral dx
\newcommand{\dx}{\mathrm{d}x}

% Alias for the Solution section header
\newcommand{\solution}{\textbf{\large Solution}}

% Probability commands: Expectation, Variance, Covariance, Bias
\newcommand{\E}{\mathrm{E}}
\newcommand{\Var}{\mathrm{Var}}
\newcommand{\Cov}{\mathrm{Cov}}
\newcommand{\Bias}{\mathrm{Bias}}

\newcommand*\circled[1]{\tikz[baseline=(char.base)]{
            \node[shape=circle,draw,inner sep=2pt] (char) {#1};}}

\newcommand\myeqq{\stackrel{\mathclap{\normalfont\mbox{\scriptsize\text{set}}}}{=}}

\makeatother

\begin{document}

\maketitle

\pagebreak

\begin{homeworkProblem}
a) What is the radius of convergence of the Taylor series of the function $z \operatorname{coth} z$ at $z=0$?

b) What is the radius of convergence of the Taylor series of the function $\left(z^{3}+1\right) /\left(z^{4}+1\right)$ at $z=1$?\\

\solution\\

\part\\

Let $f(z) = z \,\text{coth}\, z$. The radius of convergence of $f(z)$ centered at $z = 0$ is the distance from $z$ to the nearest singularity $z_0$. Write $f(z)$ as
\begin{align*}
 f(z) = \frac{z\, \text{cosh}\, z}{\text{sinh}\, z}
 ,\end{align*}
and note that $\text{sinh}\, z$ has zeroes $k \pi i$ for $k \in \mathbb{Z}$ and $z \cosh{z}$ has only one zero at $z = 0$. So we compute
\begin{align*}
 \lim_{z \to 0}(z-0)f(z) = \lim_{z \to 0} \frac{z^2 \cosh{z}}{\sinh{z}} = 0
 ,\end{align*}
so at $z =0$, $f$ is actually analytic and thus has a removable singularity there. Let us now check the other singularities, so compute
\begin{align*}
  \lim_{z \to i \pi}(z- i \pi)f(z) = \lim_{z \to i \pi} z(z- i\pi) \frac{\cosh{z}}{\sinh{z}} \stackrel{u \, \mapsto z - i\pi}{=} \lim_{u \to 0} u(u + i \pi) \frac{\cosh(u + i \pi)}{\sinh(u + i\pi)}
 .\end{align*}
Using the periodicity of $\sinh$ and $\cosh$ then gives us
\begin{align*}
  \lim_{z \to i \pi}(z- i \pi)f(z) = \lim_{u \to 0} \frac{u^2 \cosh{u}}{\sinh{u}} + i\pi \lim_{u \to 0} \frac{u \cosh{u}}{\sinh{u}} = i\pi
 .\end{align*}
Since this limit exists and is finite an nonzero, $f$ has simple poles for every non-zero integer multiple of $i\pi$. So, the closest poles are $\pm i \pi$, which are in fact equidistant from $z = 0$ of distance $\pi$, so the radius of convergence is $R = \pi$.\\

\part\\

Let $f(z) = \frac{z^3 + 1}{z^4 + 1}$. Like \textbf{Part A}, the radius of convergence of $f(z)$ centered at $z =1$ is the distance from $z$ to the nearest singularity. So write $f(z)$ as
\begin{align*}
 f(z) = \frac{z^3 + 1}{z^4 + 1}
 .\end{align*}
Factoring the denominator gives $z^4 + 1 = \left(z - \frac{1 + i}{\sqrt{2}}\right) \left(z - \frac{1 - i}{\sqrt{2}}\right) \left(z - \frac{-1 + i}{\sqrt{2}}\right) \left(z - \frac{-1 - i}{\sqrt{2}}\right)$, so it has zeroes
\begin{align*}
 z_1 = \frac{1}{\sqrt{2}} + i \frac{1}{\sqrt{2}}, z_2 = \frac{1}{\sqrt{2}} - i \frac{1}{\sqrt{2}}, z_3 = -\frac{1}{\sqrt{2}} + i \frac{1}{\sqrt{2}}, z_4 = - \frac{1}{\sqrt{2}} - i \frac{1}{\sqrt{2}}
 .\end{align*}
Factoring the numerator gives $z^3 + 1 = (z+1)(z^2 - z + 1)$, so it has zeroes at
\begin{align*}
 z_5 = -1, z_6 = \frac{1}{2} + i\frac{\sqrt{3}}{2}, z_7 = \frac{1}{2} - i\frac{\sqrt{3}}{2}
 .\end{align*}
Hence, the numerator is continuous and holomorphic at $z_1, z_2, z_3,$ and $z_4$, so $f$ has simple poles at those points, of which $z_1$ and $z_2$ are closest since
\begin{align*}
 \abs{1 - z_1} = \abs{1 - z_2} = \sqrt{2 - \sqrt{2}} < \sqrt{2 + \sqrt{2}} = \abs{1 - z_3} = \abs{z - z_4}
 ,\end{align*}
and hence the radius of convergence is $R = \sqrt{2 - \sqrt{2}}$.
\end{homeworkProblem}

\pagebreak

\begin{homeworkProblem}
  Locate the poles of the function $\frac{z}{\sin ^{2} z}$, and calculate its principal parts and residues.\\

  \solution\\

  Let $f(z) = \frac{z}{\sin^2{z}}$. The numerator $z$ has a zero at $z = 0$ and the denominator $\sin^2{z}$ has zeroes $z = k \pi$ for $k \in \mathbb{Z}$. Now, we compute
  \begin{align*}
   \lim_{z \to 0}(z-0)f(z) = \lim_{z \to 0} \frac{z^2}{\sin^2{z}} = \lim_{z \to 0} \left( \frac{z}{\sin^2{z}} \right)^2 = 1^2 = 1,
   \end{align*}
  so since this limit exists and is nonzero and finite, $f$ has a simple pole at $z = 0$ with residue $\text{Res}[f, 0] = 1$. At $z = k\pi$ for nonzero $k \in \mathbb{Z}$, we have
  \begin{align*}
   \lim_{z \to k \pi} \frac{d}{dz}\left[(z-k \pi)^2 \frac{z}{\sin^2{z}}\right] &\stackrel{u\, \mapsto z - k\pi}{=} \lim_{u \to 0} \frac{d}{du} \left[ u^2 \cdot \frac{u + k\pi}{\sin^2(u + k \pi)} \right]\\
   &= \lim_{u \to 0} \frac{-u[2u(u + k \pi)\cos{u} - (3u + 2k \pi)\sin{u}]}{\sin^3{u}}\\
   &= \lim_{u \to 0} \frac{-u^2(u + k\pi) + u^2(3u + 2k\pi)}{u^3}\\
   &= 1
   .\end{align*}
Since this limit exists and is nonzero and finite, we have that $f$ has a pole of order 2 at $z = k \pi$ for nonzero integers $k$ with residue $\text{Res}[f, k \pi] = 1 = b_1$. To find the $b_2$ term in this Laurent series, simply note that we have
  \begin{align*}
    f(z) = \frac{z}{\sin^2{z}} = \frac{1}{z - k\pi} + \frac{b_2}{(z-k \pi)^2} + \cdots
   ,\end{align*}
  which multiplying through by $(z-k \pi)^2$ and taking the limit as $z \to k \pi$ gives
  \begin{align*}
   b_2 = \lim_{z \to k\pi}(z-k\pi)^2 \frac{z}{\sin^2{z}} = k \pi
   .\end{align*}
   In summary, expanding $f$ around $z = 0$ gives
  \begin{align*}
   f(z) = \frac{z}{\sin^2{z}} = \underbrace{\frac{1}{z}}_{\text{principal}} + \cdots
   .\end{align*}
  Expanding around $z = k\pi$ for any nonzero $k$ gives
  \begin{align*}
   f(z) = \frac{z}{\sin^2{z}} = \underbrace{\frac{k\pi}{(z-k\pi)^2} + \frac{1}{z - k\pi}}_{\text{principal}} + \cdots
   .\end{align*}
\end{homeworkProblem}

\pagebreak

\begin{homeworkProblem}
  \textbf{Gamelin \textsection V.7 Exercise 9}: Show that if the analytic function $f(z)$ has a zero of order $N$ at $z_{0}$, then $f(z)=g(z)^{N}$ for some function $g(z)$ analytic near $z_{0}$ and satisfying $g^{\prime}\left(z_{0}\right) \neq 0$.\\

  \solution

   \begin{proof}
     If $f$ has a zero at $z = z_0$ of order $N$, we can write $f$ as
     \begin{align*}
      f(z) = (z-z_0)^N\tilde{f}(z)
      ,\end{align*}
    where $\tilde{f}(z_0) \neq 0$ and has a convergent power series. I claim that we can take
    \begin{align*}
     g(z) = (z-z_0)e^{ \frac{\ln{\tilde{f}(z)}}{N} }
     \end{align*}
    such that $f(z) = g(z)^N$. Indeed, note that by construction of our $g$ and for a suitable choice of branch for the logarithm, we have
    \begin{align*}
     f(z) = g(z)^N = (z-z_0)^Ne^{\ln{\tilde{f}(z)}} = (z-z_0)^N\tilde{f}(z)
     .\end{align*}
    Further,
    \begin{align*}
     g'(z) = e^{ \frac{\ln{\tilde{f}(z)}}{N} } + (z-z_0) \left(\frac{\tilde{f}'(z)}{\tilde{f}(z)}\right)e^{ \frac{\ln{\tilde{f}(z)}}{N} } = e^{ \frac{\ln{\tilde{f}(z)}}{N} }\left(1 + (z-z_0)\frac{\tilde{f}'(z)}{\tilde{f}(z)}\right)
     ,\end{align*}
    so we have $g'(z_0) = e^{ \frac{\ln{\tilde{f}(z)}}{N} }$ since $\tilde{f}(z_0) \neq 0$. The exponential is always greater than zero, so indeed we have $g'(z_0) \neq 0$, and our constructed $g$ works.
   \end{proof}
\end{homeworkProblem}
\end{document}
