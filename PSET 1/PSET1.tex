\documentclass{article}
\usepackage[usenames, dvipsnames]{color}
\usepackage{fancyhdr}
\usepackage{extramarks}
\usepackage{amsmath}
\usepackage{amssymb}
\usepackage{mathtools}
\usepackage{amsthm}
\usepackage{amsfonts}
\usepackage[per-mode=fraction]{siunitx}
\usepackage{tikz}
\usepackage{graphicx}
\usepackage{float}
\usepackage{afterpage}
\usepackage{systeme}
\usepackage{makecell}
\usepackage{nicematrix}
\usepackage{physics}
\usepackage{float}
\usepackage{tikz}
\usetikzlibrary{math,arrows,positioning,shapes,fit,calc}
\usepackage{dsfont}
\usepackage{wrapfig}
\usepackage{pgfplots}
\usepackage{esint}
\pgfplotsset{compat=1.8}

\definecolor{mediumgreen}{RGB}{0,153,0}
\newtheorem{theorem}{Theorem}[section]
\newtheorem{lemma}[theorem]{Lemma}
\usetikzlibrary{trees}

%
% Basic Document Settings
%

\topmargin=-0.45in
\evensidemargin=0in
\oddsidemargin=0in
\textwidth=6.5in
\textheight=9.0in
\headsep=0.25in

\linespread{1.1}

\pagestyle{fancy}
\lhead{}
\chead{\hmwkClass\ (\hmwkClassInstructor\/): \hmwkTitle}
\rhead{\firstxmark}
\lfoot{\lastxmark}
\cfoot{\thepage}

\renewcommand\headrulewidth{0.4pt}
\renewcommand\footrulewidth{0.4pt}

\setlength\parindent{0pt}

%
% Create Problem Sections
%

\newcommand{\enterProblemHeader}[1]{
    \nobreak\extramarks{}{Problem \arabic{#1} continued on next page\ldots}\nobreak{}
    \nobreak\extramarks{Problem \arabic{#1} (continued)}{Problem \arabic{#1} continued on next page\ldots}\nobreak{}
}

\newcommand{\exitProblemHeader}[1]{
    \nobreak\extramarks{Problem \arabic{#1} (continued)}{Problem \arabic{#1} continued on next page\ldots}\nobreak{}
    \stepcounter{#1}
    \nobreak\extramarks{Problem \arabic{#1}}{}\nobreak{}
}

\setcounter{secnumdepth}{0}
\newcounter{partCounter}
\newcounter{homeworkProblemCounter}
\setcounter{homeworkProblemCounter}{1}
\nobreak\extramarks{Problem \arabic{homeworkProblemCounter}}{}\nobreak{}

%
% Homework Problem Environment
%
% This environment takes an optional argument. When given, it will adjust the
% problem counter. This is useful for when the problems given for your
% assignment aren't sequential. See the last 3 problems of this template for an
% example.
%
\newenvironment{homeworkProblem}[1][-1]{
    \ifnum#1>0
        \setcounter{homeworkProblemCounter}{#1}
    \fi
    \section{Problem \arabic{homeworkProblemCounter}}
    \setcounter{partCounter}{1}
    \enterProblemHeader{homeworkProblemCounter}
}{
    \exitProblemHeader{homeworkProblemCounter}
}

%
% Homework Details
%   - Title
%   - Due date
%   - Class
%   - Section/Time
%   - Instructor
%   - Author
%

\newcommand{\hmwkTitle}{Homework 1}
\newcommand{\hmwkDueDate}{Sunday January 9, 2022}
\newcommand{\hmwkClass}{Math 382}
\newcommand{\hmwkClassTime}{Section A}
\newcommand{\hmwkClassInstructor}{Prof. Ezra Getzler}
\newcommand{\hmwkAuthorName}{\textbf{Anthony Tam}}

%
% Title Page
%

\title{
    \vspace{2in}
    \textmd{\textbf{\hmwkClass:\ \hmwkTitle}}\\
    \normalsize\vspace{0.1in}\small{Due\ on\ \hmwkDueDate\ at 5:00 PM}\\
    \vspace{0.1in}\large{\textit{\hmwkClassInstructor\ }}
    \vspace{3in}
}

\author{\hmwkAuthorName}
\date{}

\renewcommand{\part}[1]{\textbf{\large Part \Alph{partCounter}}\stepcounter{partCounter}\\}

%
% Various Helper Commands
%

% For derivatives
\newcommand{\deriv}[1]{\frac{\mathrm{d}}{\mathrm{d}x} (#1)}

% For partial derivatives
\newcommand{\pderiv}[2]{\frac{\partial}{\partial #1} (#2)}

% Integral dx
\newcommand{\dx}{\mathrm{d}x}

% Alias for the Solution section header
\newcommand{\solution}{\textbf{\large Solution}}

% Probability commands: Expectation, Variance, Covariance, Bias
\newcommand{\E}{\mathrm{E}}
\newcommand{\Var}{\mathrm{Var}}
\newcommand{\Cov}{\mathrm{Cov}}
\newcommand{\Bias}{\mathrm{Bias}}

\newcommand*\circled[1]{\tikz[baseline=(char.base)]{
            \node[shape=circle,draw,inner sep=2pt] (char) {#1};}}

\newcommand\myeqq{\stackrel{\mathclap{\normalfont\mbox{\scriptsize\text{set}}}}{=}}

\makeatother

\begin{document}

\maketitle

\pagebreak

\begin{homeworkProblem}
  Show that the triangle with vertices $0, z$ and $w$ is equilateral if and only if $|z|^{2}=|w|^{2}=2 \operatorname{Re}(\bar{z} w)$.\\

  \solution

  \begin{proof}
    The three sides of the triangle formed by $0, z, \text{ and } w$ considered as points in $\mathbb{C}$ are the line segments connecting $0$ and $z$, $0$ and $w$, and $z$ and $w$. It follows then that the lengths of these sides are then $\abs{z - 0} = \abs{z}$, $\abs{w - 0} = \abs{w}$, and $\abs{z - w}$ respectively. Note that we could have just as easily chosen $\abs{0 - z}$, $\abs{0 - w}$, and $\abs{w - z}$ as representatives of the length. So, the equivalent statement to prove is
  \begin{align*}
   \abs{z} = \abs{w} = \abs{ z - w} \text{ if and only if } \abs{z}^2 = \abs{w}^2 = 2 \text{Re}(\overline{z}w)
   .\end{align*}
  Suppose then that $\abs{z} = \abs{w} = \abs{ z - w}$. This is true if and only if by squaring everything,
  \begin{align*}
    \abs{z}^2 = \abs{w}^2 &= \abs{ z - w}^2\\
    &= (z- w)\overline{(z-w)}\\
    &= (z-w)(\overline{z} - \overline{w})\\
    &= z\overline{z}- w\overline{z} - \overline{w}z + w\overline{w}\\
    &= \abs{z}^2 + \abs{w}^2 - w\overline{z} - \overline{w}z\\
    &= \abs{z}^2 + \abs{w}^2 - 2 \text{Re}(\overline{z}w)
   ,\end{align*}
  where in the last line we used the fact that $\text{Re}(z) = \frac{1}{2}(z + \overline{z})$ and that $\overline{\overline{z}w} = \overline{\overline{z}} \, \overline{w} = z\overline{w}$. So overall, we have
  \begin{align*}
     \abs{z} = \abs{w} = \abs{ z - w} &\iff \abs{z}^2 = \abs{w}^2 = \abs{z}^2 + \abs{w}^2 - 2 \text{Re}(\overline{z}w) \\
     &\iff \abs{z}^2 = \abs{w}^2 = 2 \text{Re}(\overline{z}w)
   ,\end{align*}
  as claimed.
  \end{proof}
\end{homeworkProblem}

\pagebreak

\begin{homeworkProblem}
  Let $n$ be an integer. Show that the function $\cos n z$ is a polynomial in the function $\cos z$. For example, $\cos 2 z=2(\cos z)^{2}-1$.\\

  \solution

  \begin{proof}
    Recall that the complex exponential $e^{inz}$ for $n \in \mathbb{Z}$ can be rewritten as
    \begin{align*}
     e^{inz} = \cos{nz} + i\sin{nz}
     ,\end{align*}
    so in particular $\text{Re}(e^{inz}) = \cos{nz}$. To rewrite cosine then, we expand the following complex binomial using the binomial formula:
    \begin{align*}
     \cos{nz} &= \text{Re}(e^{inz})\\
     &= \text{Re}((e^{iz})^n)\\
     &= \text{Re}((\cos{z} + i \sin{z})^n)\\
     &= \text{Re} \left[ \sum_{k= 0}^{n} \left(\begin{matrix} n \\ k \end{matrix}\right) \cos^{n-k}z(i^k\sin^kz) \right]
     .\end{align*}
    Note now that $i^k$ is real if and only if $k$ is even, so make a substitution $k \mapsto 2m$. Summing over even indicies is equivalent to taking a sum from $m = 0$ to $m = \lfloor n/2 \rfloor$, where $\lfloor \cdot \rfloor$ denotes the floor function which returns the greatest integer less than or equal to the argument. To see why this is, note that if $n$ is even, then $\lfloor n/2 \rfloor = n/2$ so the last term is $2\cdot n/2$, but if $n$ is odd, then $\lfloor n/2 \rfloor = (n-1)/2$ so the last term is $2\cdot (n-1)/2 = n-1$. Further, $i^{2m}$ will oscillate between $i^2 = -1$ and $i^4 = 1$, so all together we have
    \begin{align*}
     \cos{nz} &= \sum_{m=0}^{\lfloor n/2 \rfloor}\left(\begin{matrix} n \\ 2m \end{matrix}\right) \cos^{n-2m}z\sin^{2m}z (-1)^m\\
     &= \sum_{m=0}^{\lfloor n/2 \rfloor}\left(\begin{matrix} n \\ 2m \end{matrix}\right) \cos^{n-2m}z(\sin^{2}z)^m (-1)^m\\
     &= \sum_{m=0}^{\lfloor n/2 \rfloor}\left(\begin{matrix} n \\ 2m \end{matrix}\right) \cos^{n-2m}z(1- \cos^2 z)^m (-1)^m\\
     &= \sum_{m=0}^{\lfloor n/2 \rfloor}\left(\begin{matrix} n \\ 2m \end{matrix}\right) \cos^{n-2m}z(\cos^2 z - 1)^m
     ,\end{align*}
    where we used the identity $\sin^2z + \cos^2 z = 1$ in the second to last equality. Note that the highest power of cosine in the binomial in parentheses will be $2m$, which times the cosine factor outside of the parentheses will give us a polynomial in $\cos z$, in fact of degree $n - 2m + 2m = n$.
  \end{proof}
\end{homeworkProblem}

\pagebreak

\begin{homeworkProblem}
\textbf{Gamelin Exercise I.8.5.} Let $S$ denote the two slits along the imaginary axis in the complex plane, one running from $i$ to $+i \infty$, the other from $-i$ to $-i \infty$. Show that $(1+i z) /(1-i z)$ lies on the negative real axis $(-\infty, 0]$ if and only if $z \in S$. Show that the principal branch
$$
\operatorname{Tan}^{-1} z=\frac{1}{2 i} \text{Log} \left(\frac{1+i z}{1-i z}\right)
$$
maps the slit plane $\mathbb{C} \backslash S$ one-to-one onto the vertical strip $\{|\text{Re}\;w|<$ $\pi / 2\}$.\\

\solution

\begin{proof}
  Let $ \frac{1 + iz}{1-iz} = v$, where $v \in \mathbb{C}$. We want to show that $v \in (-\infty, 0]$ if and only if $z \in S$. An easy way to show this is to construct a one-to-one mapping from the sections of interest in the $z$-plane to the desired sections of the $v$-plane. First solving for $z$ in $v$ by clearing the denominator gives
  \begin{align*}
   1 + iz &= v - ivz \\
    iz + ivz &= v-1 \\
    iz(v+1) &= v-1 \\
    \iff z = \frac{1}{i} \frac{v-1}{v+1} &= i \frac{1 - v}{1+v}
   ,\end{align*}
  which implies that this is a one-to-one mapping. Now we compute the real limits of the right hand side of the last equality to draw some conclusions about $z$. Take careful note that we are only taking real values of $v$ and carefully choosing straight line paths along the real-axis, so we evaluate the limits in the usual real-valued function way:
  \begin{align*}
   \lim_{v \to -\infty} \frac{1-v}{1+v} = -1 \;, \; \lim_{v \to -1^{-}} \frac{1-v}{1+v} \text{ tends to}\, -\infty,\\
   \lim_{v \to -1^{+}} \frac{1-v}{1+v} = \text{ tends to}\, +\infty, \;\text{ and }\; \lim_{v \to 0} \frac{1-v}{1+v} = 1
   .\end{align*}
  Hence, $\frac{1-v}{1+v}i$ goes from $-i$ to $-i\infty$ along the $\mathbf{Im}$-axis as $v$ goes from $-\infty$ to $-1^{-}$. Further, $\frac{1-v}{1+v}i$ goes from $i$ to $i \infty$ along the \textbf{Im}-axis as $v$ goes from $-1^{+}$ to $0$. Since this map is one-to-one, there is a correspondence between the subsets
  \begin{align*}
   \underbrace{[-i, -i\infty)}_{\subset \, z-\text{plane}} \longleftrightarrow \underbrace{(-\infty, -1^{-}]}_{\subset \, v-\text{plane}} \; \text{ and } \underbrace{[i, i\infty)}_{\subset \, z-\text{plane}} \longleftrightarrow \underbrace{[-1^{+}, 0]}_{\subset \, v-\text{plane}}
   .\end{align*}
  Note that this is slight abuse of notation with $-1^{+}$ and $-1^{-}$ in the set notation, but the important remark here is that the union of the subsets of the $v$-plane give $(-\infty, 0]$ and the union of the subsets of the $z$-plane give $S$ so that $z \in S$ if and only if $v \in (-\infty, 0]$ by this one-to-one mapping as claimed.\\

  To show the next claim, it is best to build up the function in question as the composition of one-to-one mappings, where we use the fact that the composition of two one-to-one maps is another one-to-one map. We have already determined that $v = \frac{1 + iz}{1-iz}$ is one-to-one from $\mathbb{C}\setminus S \rightarrowtail \mathbb{C}\setminus (-\infty, 0]$. So, let $u = \text{Log}\;v$ and note that now this is a one-to-one (and onto) map from $\mathbb{C}\setminus (-\infty, 0] \rightarrowtail \left\{ \abs{\text{Im}\,u} < \pi \right\}$ since Log is defined to be a single valued inverse for $e^{v}$. Let $w = \frac{1}{2i} u$ and then remark that this composition now maps $\left\{ \abs{\text{Im}\,u} < \pi \right\} \rightarrowtail \left\{ \abs{\text{Re}\,w} < \frac{\pi}{2} \right\}$ in a one-to-one manner. To summarize,
  \begin{align*}
    \mathbb{C}\setminus S \stackrel{\mathclap{v}}{\rightarrowtail} \mathbb{C}\setminus (-\infty, 0] \stackrel{\mathclap{u}}{\rightarrowtail} \left\{ \abs{\text{Im}\,u} < \pi \right\} \stackrel{\mathclap{w}}{\rightarrowtail} \left\{ \abs{\text{Re}\,w} < \pi/2 \right\}
   ,\end{align*}
  with $\text{Tan}^{-1}z = w \circ u \circ v : \mathbb{C}\setminus S \rightarrowtail \left\{ \abs{\text{Re}\,w} < \frac{\pi}{2} \right\}$ a one-to-one map.
\end{proof}
\end{homeworkProblem}

\end{document}
